  \item \textbf{Year 3:} \euro{}\,147\,900 / month (annual burn \euro{}\,1\,775\,000)
\end{itemize}

\section{Revenue Projection Model}
To estimate revenue, we use a model based on prudent yet ambitious assumptions regarding pricing and customer acquisition rates.

\subsection{Model Assumptions}
\begin{enumerate}
    \item \textbf{Pricing (ARPA - Average Revenue Per Account):}
    \begin{itemize}
        \item \textbf{Pro Plan (SaaS):} An average value per customer of \textbf{\euro{} 300/month}.
        \item \textbf{Enterprise Plan (On-Premise):} An Annual Contract Value (ACV) of \textbf{\euro{} 18,000}, which translates to \textbf{\euro{} 1,500 MRR} per customer.
    \end{itemize}

    \item \textbf{Net New Customer Acquisition Rate:}
    \begin{itemize}
        \item \textbf{Year 1:} Average of \textbf{3 new Pro customers} and \textbf{0.33 Enterprise customers} per month (4 Enterprise contracts/year).
        \item \textbf{Year 2:} Average of \textbf{8 new Pro customers} and \textbf{0.75 Enterprise customers} per month (9 Enterprise contracts/year).
        \item \textbf{Year 3:} Average of \textbf{15 new Pro customers} and \textbf{1.5 Enterprise customers} per month (18 Enterprise contracts/year).
        \item \textbf{Year 4:} Average of \textbf{25 new Pro customers} and \textbf{2 Enterprise customers} per month (24 Enterprise contracts/year).
    \end{itemize}

    \item \textbf{Churn Rate:}
    \begin{itemize}
        \item A monthly churn rate of \textbf{2\%} for Pro customers.
        \item An annual churn rate of \textbf{1\%} for Enterprise customers (assuming high-stickiness annual contracts).
    \end{itemize}
\end{enumerate}

\subsection{Market benchmarks and customer-acquisition rationale}

Our customer acquisition model is based on conservative assumptions drawn from established B2B SaaS industry benchmarks. For our product-led growth motion, we assume a free-to-paid conversion rate that lies at the cautious end of the typical performance spectrum for freemium products.

Retention assumptions are similarly prudent. Our projected monthly churn rates for paying customers are aligned with those of strong, but not exceptional, B2B SaaS operators. For enterprise clients, where contracts are longer and relationships are deeper, we assume a significantly lower annual churn rate, mirroring the high "stickiness" observed in best-in-class, publicly-traded infrastructure software companies.

The productivity targets for our enterprise sales team are also set conservatively within the standard performance envelope for an Account Executive in the enterprise software space. We project a number of annual deals per salesperson that is well within industry norms, especially when supported by a flow of qualified leads from our product-led funnel.

Taken together, these deliberately restrained assumptions ensure that the acquisition curve in our financial model is plausible and not reliant on best-case-scenario performance.


\subsection{Break-Even Projection}

\begin{table}[H]
\centering
\caption{Break-even projection with moderate churn assumptions}
\label{tab:break_even_moderate_churn}
\begin{tabularx}{\textwidth}{@{}l c c >{\raggedleft\arraybackslash}X
                                    >{\raggedleft\arraybackslash}X
                                    >{\raggedleft\arraybackslash}X@{}}
\toprule
\textbf{End of period} &
\textbf{Pro cust.} &
\textbf{Ent.\ cust.} &
\textbf{Estimated MRR} &
\textbf{Monthly cost} &
\textbf{Deficit / surplus} \\
\midrule
End Year 1  & $\sim$11  & 6  & \euro{}\,12\,200  & \euro{}\,59\,600  & \textbf{–}\,47\,400 \\
End Year 2  & $\sim$41  & 18 & \euro{}\,39\,100  & \euro{}\,91\,250  & \textbf{–}\,52\,100 \\
End Year 3  & $\sim$97  & 42 & \euro{}\,91\,600  & \euro{}\,147\,900 & \textbf{–}\,56\,300 \\
Mid-Year 4  & $\sim$143 & 59 & \euro{}\,131\,800 & \euro{}\,165\,000 & \textbf{–}\,33\,200 \\
End Year 4  & $\sim$183 & 77 & \euro{}\,171\,000 & \euro{}\,165\,000 & \textbf{+}\,6\,000  \\
\bottomrule
\end{tabularx}
\end{table}

\subsection{Conclusion}
Based on this aggressive but plausible growth model, the operational break-even point will likely be reached **during the fourth year of activity**.

\subsubsection*{Strategic Implications for Investors}
\begin{itemize}
    \item \textbf{Focus on Growth, not Short-Term Profitability:} This plan is consistent with a Venture Capital-backed strategy, where the goal of the initial funding rounds is to capture significant market share, not to achieve immediate sustainability.
    \item \textbf{Importance of Key Metrics:} The validity of this projection is entirely dependent on the team's ability to achieve the hypothesized acquisition and retention metrics. MVP KPIs (Activation Rate, 1-Month Retention) will be crucial to demonstrate that the growth engine is working as planned.
    \item \textbf{Need for Future Funding:} The plan highlights the necessity of at least one subsequent funding round (Seed/Series A) towards the end of Year 2 / beginning of Year 3 to finance the scaling phase and reach the break-even point.
\end{itemize}
In summary, the model demonstrates a path to long-term sustainability but also underscores the capital-intensive nature of a strategy that aims to build a market leader in a competitive sector.


\subsection{Revenue Streams}

