\section{Break-Even Analysis}
\subsection{Overview of the financial simulation model}
\label{subsec:simple-model}

This subsection explains the model in plain language. It describes what goes in, what comes out, and the small set of rules the simulation follows month by month.

\paragraph{What the model answers:}
It estimates, over 5 years (60 months): monthly revenues, costs, net profit/loss, cash balance, runway, and the month when we reach break-even. It also separates PLG (self-service) and SLG (enterprise) revenues.

\paragraph{Main inputs:}
\begin{itemize}
\item \textbf{Starting cash:} the initial capital available at month 0.
\item \textbf{Costs per category and year:} Management, R\&D, PLG, SLG, Partner, Infrastructure, G\&A. Each category has an \emph{uplift} multiplier to account for prudential overhead (compliance, legal, recruiting, cloud, etc.).
\item \textbf{Prices:} monthly list prices for Standard, Business, Scale, and an average monthly enterprise MRR (ACV/12). Add-on pods have a unit price.
\item \textbf{Customer dynamics:} monthly PLG acquisitions per tier for each year; yearly enterprise (SLG) deals; churn rates (PLG monthly, SLG annual converted to monthly).
\item \textbf{Add-ons and store:} per-tier adoption rate and average pods; store one-off revenue assumptions (rate, items, average price).
\item \textbf{Policy gates (optional):} cost cut when runway is low, and temporary reduction of PLG acquisitions when runway is very low.
\end{itemize}

\paragraph{Monthly cycle (what happens each month):}
\begin{enumerate}
\item \textbf{Apply churn at the start of the month.} We reduce the active customer base by the churn rate (PLG monthly; SLG uses the monthly equivalent of the annual churn).
\item \textbf{Add new customers.} For PLG, we add the planned new sign-ups for the current year (possibly reduced by a soft-freeze multiplier). For SLG, new enterprise contracts are spread quarterly (one quarter of the annual plan every 3rd month).
\item \textbf{Compute recurring revenue.} PLG and SLG subscription MRR come from actives times price. Add-on MRR is a share of actives (adoption rate) times average pods times pod price.
\item \textbf{Add store revenue (one-off).} A fraction of total actives purchase items from the store; this is treated as non-recurring cash in the month.
\item \textbf{Compute monthly costs.} We apply the per-category uplifted annual cost divided by 12, then a policy multiplier if a spend cut is active.
\item \textbf{Compute P\&L and cash.} Net result for the month is total revenue (recurring + store) minus costs. Cash balance increases (or decreases) by this net result.
\item \textbf{Update burn and runway.} Burn is the positive part of monthly losses. Runway is cash divided by the moving-average burn over the last \emph{W} months (default 3). If there is no burn, runway is infinite.
\item \textbf{Set policy for next month.} If runway is below thresholds, the model can reduce discretionary costs and/or slow paid PLG acquisitions in the following month.
\end{enumerate}

% \paragraph{What the charts show}
% \begin{itemize}
% \item A stacked area of \textbf{PLG total MRR} (plans + add-ons) vs. \textbf{SLG MRR}, with the \textbf{monthly cost} line overlaid.
% \item A secondary axis showing the \textbf{runway (months)}.
% \item Vertical markers for \textbf{operational break-even} (recurring MRR  monthly costs) and \textbf{cash break-even} (net result  0 including store).
% \end{itemize}

\paragraph{Key definitions (kept intuitive):}
\begin{description}
\item[Operational break-even] First month when recurring MRR (subscriptions + add-ons) is at least equal to monthly operating costs.
\item[Cash break-even] First month when total revenue (recurring + store one-offs) covers mon\-th\-ly costs, i.e., net result .
\item[Runway] How many months of operations remain at the current burn rate; we use a short moving average of recent burn to make it more stable.
\end{description}

\paragraph{Why this is conservative but controllable:}
Uplift multipliers inflate costs in a credible way, capturing overhead that often appears in real execution. Policy gates make the plan self-correcting when runway tightens, protecting the path to break-even without assuming unrealistic growth.

\subsection{Financial Model: Variables and Notation}
\begin{itemize}
  \item Time is in months: $t = 1,2,\dots,T$ with $T = 12 \cdot \text{years}$. Year index $y(t) = \lceil t/12 \rceil$.
  \item Plans: PLG tiers $p \in \mathcal{P}=\{\text{standard},\ \text{business},\ \text{scale}\}$ and enterprise (SLG).
  \item Customers: $c^p_t$ (active PLG customers of tier $p$ at month $t$), $c^{\mathrm{ent}}_t$ (active enterprise).
  \item Prices: $P_p$ (monthly price for tier $p$), $P_{\mathrm{ent}}$ (enterprise MRR), $P_{\mathrm{pod}}$ (add-on pod).
  \item Costs: $\mathrm{Cost}_{k,y}$ annual base cost for category $k$ in year $y$ with uplift factor $u_k$.
  \item Churn: $\delta_{\mathrm{PLG}}$ (monthly), $\delta^{\mathrm{annual}}_{\mathrm{SLG}}$ (annual). 
  \item Add-ons: adoption rate $r_p$, average pods $\bar{n}_p$.
  \item Store: purchase rate $\rho$, average purchases $\bar{u}$, average sale price $P_{\mathrm{store}}$.
  \item Policy parameters: spend cut $\gamma$ (e.g., $0.10$), runway guardrail $R_{12}$ (12 months), soft-freeze $R_{9}$ (9 months), window $W$ (e.g., $3$ months).
\end{itemize}

\newpage
\subsection{Cost Model}
\paragraph{Annual cost with per-category uplift.}
\begin{align}
C^{\mathrm{annual}}_y 
  &= \sum_{k \in \{\mathrm{MGMT},\mathrm{RND},\mathrm{PLG},\mathrm{SLG},\mathrm{PARTNER},\mathrm{INFRA},\mathrm{GA}\}}
     \mathrm{Cost}_{k,y}\, u_k.
\end{align}

\paragraph{Monthly operating cost with spend gate multiplier.}
Let $s_t$ be the spend multiplier (policy below). Then:
\begin{align}
C^{\mathrm{monthly}}_t 
  &= \frac{C^{\mathrm{annual}}_{y(t)}}{12} \cdot s_t.
\end{align}

\subsection{Customer Dynamics}
\paragraph{Churn conversion (enterprise annual $\to$ monthly).}
\begin{align}
