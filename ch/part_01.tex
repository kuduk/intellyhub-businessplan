\documentclass[11pt, a4paper, oneside]{article}

% --- PACHETTI NECESSARI ---
\usepackage{graphicx} % Per includere immagini (logo)
% Modificati i margini per dare più spazio all'intestazione
\usepackage[a4paper, top=4cm, bottom=2.5cm, left=2.5cm, right=2.5cm, headheight=1.2cm, headsep=1.5cm]{geometry}
\usepackage{xcolor} % Per definire e usare colori personalizzati
\usepackage{titlesec} % Per personalizzare i titoli delle sezioni
\usepackage{enumitem} % Per personalizzare le liste
\usepackage{hyperref} % Per creare link interni ed esterni
\usepackage{ragged2e} % Per un migliore allineamento del testo
\usepackage{lettrine} % Per le lettere iniziali
\usepackage{fancyhdr} % Per header e footer personalizzati
\usepackage{tabularx} % Per tabelle con larghezza definita
\usepackage{amsfonts} % Per simboli matematici se necessari
\usepackage{amsmath}
\usepackage[utf8]{inputenc}
\usepackage{graphicx}
\usepackage{booktabs}
\usepackage{tikz}
\usepackage{pgfplots}
\usepackage{float}
\usepackage{eurosym}
\usepackage{microtype}
\usepackage{siunitx} 
\pgfplotsset{compat=1.18}
% --- IMPOSTAZIONE FONT E LINGUA (Richiede XeLaTeX) ---
\usepackage{fontspec}
\usepackage{xeCJK}
\usepackage{multirow}
\usepackage{booktabs,longtable,siunitx,ragged2e,placeins} 

\def\UrlBreaks{\do\.\do\/\do\-\do\_\do\?\do\&} % Per permettere le interruzioni di riga negli URL
\newcolumntype{L}{>{\raggedright\arraybackslash}X} % Per colonne a larghezza variabile con testo allineato a sinistra


% Carica i font direttamente dai file locali, specificando i diversi pesi.
% Questo è il metodo più robusto.
% Assicurati che i file .ttf siano in una sottocartella chiamata "fonts".
\setmainfont{NotoSans-Regular.ttf}[
    Path = ./fonts/,
    BoldFont = NotoSans-Bold.ttf,
    ItalicFont = NotoSans-Italic.ttf,
    BoldItalicFont = NotoSans-BoldItalic.ttf
]
\setCJKmainfont{NotoSansSC-Regular.ttf}[
    Path = ./fonts/,
    BoldFont = NotoSansSC-Bold.ttf,
    ItalicFont = NotoSansSC-Regular.ttf
]
% Definisce un nuovo font "light" per un uso personalizzato
\newfontfamily\lightfont{NotoSans-Light.ttf}[
    Path = ./fonts/,
    ItalicFont = NotoSans-LightItalic.ttf
]



% --- DEFINIZIONE COLORI DEL BRAND (Personalizzabili) ---
\definecolor{PrimaryColor}{HTML}{6A4C9C}   % Colore principale (es. Viola)
\definecolor{SecondaryColor}{HTML}{2A2F45} % Colore secondario (es. Blu Scuro)
\definecolor{AccentColor}{HTML}{8E7CC3}    % Colore d'accento
\definecolor{DarkGray}{HTML}{343a40}      % Grigio scuro per il testo

% --- IMPOSTAZIONI HYPERREF ---
\hypersetup{
    colorlinks=true,
    linkcolor=PrimaryColor,
    filecolor=AccentColor,      
    urlcolor=SecondaryColor,
    citecolor=AccentColor,
    pdftitle={Business Plan},
    pdfpagemode=FullScreen,
}

% --- PERSONALIZZAZIONE TITOLI DI SEZIONE ---
\titleformat{\section}
  {\normalfont\Large\bfseries\color{SecondaryColor}}
  {\thesection}{1em}{}
\titleformat{\subsection}
