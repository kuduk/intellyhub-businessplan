
Break-even arrives in \textbf{month 51}. At that point \textbf{recurring MRR is €247{,}103} against \textbf{€233{,}271} of costs (operational BE), and \textbf{total revenue €248{,}159} exceeds costs (cash BE). To get there the model \textbf{burns €4{,}939{,}370} in total, with a \textbf{minimum cash} of \textbf{€2{,}234{,}573} that also represents the \textbf{reserve at BE (30.9\% of the round)}. By the end of year~5, recurring MRR reaches \textbf{€326{,}645}/month ($\approx$ \textbf{€3.92M ARR}), monthly profit is \textbf{€94{,}565}, cash is \textbf{€2{,}673{,}539}, and the runway becomes effectively infinite.

\paragraph{Takeaway}
The chart makes three things obvious:
\begin{enumerate}
\item \emph{who pushes what} PLG builds the base, SLG closes the gap
\item \emph{why costs jump} deliberate capacity steps with prudential uplifts
\item \emph{how cash stays protected} runway never collapses (min 25.0 months) and accelerates as the revenue stack overtakes costs, exactly as the printed metrics report.
\end{enumerate}

\begin{table}[H]
\centering
\caption{Financial Projection Summary (Year-End)}
\label{tab:financial_summary}
\resizebox{\textwidth}{!}{
\begin{tabular}{lrrrrrrr}
\toprule
\textbf{Year End} & \textbf{Monthly Cost (€)} & \textbf{Recurring MRR (€)} & \textbf{Store Revenue (€)} & \textbf{Total Revenue (€)} & \textbf{Monthly P/L (€)} & \textbf{Cash Balance (€)} & \textbf{Runway (months)} \\
\midrule
Year 1 & 123,212 & 10,948 & 116 & 11,064 & -112,149 & 5,740,147 & 50.6 \\
Year 2 & 154,413 & 53,194 & 364 & 53,558 & -100,855 & 4,282,549 & 40.3 \\
Year 3 & 216,746 & 135,360 & 755 & 136,115 & -80,631 & 2,823,198 & 30.8 \\
Year 4 & 224,646 & 218,614 & 1,001 & 219,615 & -5,031 & 2,239,361 & 123.5 \\
Year 5 & 233,271 & 326,645 & 1,191 & 327,836 & 94,565 & 2,673,539 & Profitable \\
\bottomrule
\end{tabular}
}
\end{table}

\begin{table}[H]
\centering
\caption{Key Financial Metrics}
\label{tab:financial_metrics}
\resizebox{\textwidth}{!}{%
\begin{tabular}{ll}
\toprule
\textbf{Metric} & \textbf{Value} \\
\midrule
Operating break-even & Month 51 (Year 5): Recurring MRR €247,103 $\geq$ Costs €233,271 \\
Break-even & Month 51 (Year 5): Total Revenue €248,159 $\geq$ Costs €233,271 \\
Capital burned until cash BE & €4,939,370 \\
Monthly peak burn & €160,422 \\
Minimum cash in period & €2,210,630 (30.9\% of round) \\
Minimum runway (3-month MA) & 25.0 months \\
\bottomrule
\end{tabular}
}
\end{table}

\newpage
\section{Why CNY 59{,}829{,}055 is the right amount}

We request \textbf{CNY 59{,}829{,}055} ($\approx€7.15\text{M}$) because this is exactly the amount of capital that, in our \emph{conservative} scenario, takes the company to \textbf{operational and cash break-even in month~51} without forcing growth and while preserving a concrete safety margin. The simulation results are explicit: \textbf{cumulative burn to cash break-even = €4{,}939{,}370}, \textbf{cash on hand at break-even = €2{,}210{,}630} (i.e., \textbf{30.9\%} of the round), \textbf{minimum runway along the path = 25.0 months} (3-month moving average), and \textbf{peak monthly burn = €160{,}422}. We are asking for the amount the model shows as \emph{necessary and sufficient} to reach break-even with a structural contingency buffer.

The plan is built to be resilient: costs are not ``bare-bones'' but \textbf{prudentially uplifted} by category (infrastructure, G\&A, PLG, SLG, R\&D, management) to capture recurring items that are often underestimated (enterprise support, audits, legal, recruiting, monitoring). In addition, the model introduces \textbf{automatic guardrails}: if runway drops below 12 months, \textbf{discretionary costs are cut by 10\%} the following month; below 9 months, \textbf{paid PLG acquisitions are damped} (multiplier 0.7). These are operational rules encoded in the model, not promises. In practice, the downside is protected by mechanisms that trigger on their own.

On the revenue side, we clearly separate \textbf{PLG} (plans + add-ons) and \textbf{SLG} (enterprise). This is not cosmetic: it lets us see, month by month, where spend levers return more and rebalance without ideology. With this mix and current prices, \textbf{operational break-even} arrives in \textbf{month~51} with \textbf{recurring MRR of €247{,}103} against \textbf{monthly costs of €233{,}271}; in the same month, \textbf{cash break-even} is achieved because \textbf{total revenue (€248{,}159)} exceeds costs. By the end of year $\sim 5$, recurring MRR reaches \textbf{€326{,}645} ($\approx$ \textbf{€3.92M ARR}). In terms of capital efficiency, the \textbf{implicit burn multiple} (burn to break-even $\div$ ARR at break-even) is \textbf{~1.66x}, consistent with a conservative product+go-to-market build and with costs already uplifted credibly.

Why, then, \emph{this} amount and not less? With less capital, the model would trigger guardrails more frequently, creating operational stop-and-go (cuts/cooling periods) that stretch timelines and raise opportunity cost precisely when continuity matters most. Why not more? Because beyond this threshold the bottleneck is not budget but \textbf{channel absorption} and the natural cadence of enterprise delivery; extra cash today would increase dilution without improving outcomes relative to the model.

\textbf{Use of proceeds} remains anchored to the very categories in the simulation and their uplifts: product/R\&D (hardening, observability, security), infrastructure \& enterprise support, SLG (account, solutions/POC), PLG (content/SDK/community), partner enablement, G\&A \& compliance, management. We are not opening new spend lines: we are funding what the model already measures month by month. 

Finally, the \textbf{risk profile} is readable. Minimum runway does not fall below \textbf{25.0 months}, guardrails limit cash erosion when needed, and the \textbf{30.9\%} buffer at break-even provides headroom against procurement delays, infra/compliance spend variability, or FX moves. At the same time, the PLG/SLG separation makes it straightforward—even ex post—to show that capital allocation followed realized returns rather than a one-size-fits-all plan.

\textbf{In short:} \textbf{CNY 59{,}829{,}055} fully finances the conservative path to break-even, with sufficient buffer and automated cost discipline. It is a proportional, defensible, and—above all—\textbf{replicable} ask: investors can verify month by month that model metrics remain under control and that cash tracks the expected trajectory.

\subsection{Strategic Buffer Rationale: Navigating the AI Orchestration Frontier}
The 30.9\% capital reserve at break-even (€2.21M) represents a deliberate strategic allocation for navigating the unprecedented velocity of change in the AI orchestration market. Unlike traditional SaaS sectors where product-market fit follows predictable patterns, the AI infrastructure landscape is experiencing fundamental shifts every 3-6 months—from new LLM architectures to emerging orchestration standards like MCP. This buffer enables IntellyHub to execute rapid strategic pivots without compromising runway: whether adapting to a breakthrough in autonomous agent capabilities, integrating game-changing models that didn't exist at planning time, or shifting focus between PLG and SLG channels based on real market response. Historical precedent from successful AI infrastructure companies (Weights \& Biases, Hugging Face) demonstrates that winners in this space required 2-3 significant pivots before achieving sustainable growth—each consuming 15-20\% of available capital. Our reserve ensures we can execute at least one major strategic realignment while maintaining 12+ months of operational runway, transforming what would be existential threats into competitive advantages. This is not excess capital; it's calculated optionality insurance in a market where the only certainty is radical change, and where the ability to pivot faster than competitors—while they scramble for emergency funding—becomes the decisive factor between market leadership and obsolescence.

\newpage
\section{Go-to-Market Strategy}
% Come raggiungerai i tuoi clienti?

IntellyHub's Go-to-Market (GTM) strategy is based on a hybrid model that combines two growth engines:
\begin{enumerate}
    \item \textbf{Product-Led Growth (PLG) for SaaS:} We leverage the superiority of the product, a Free Tier, and the Automation Store to attract, activate, and convert users in a scalable, bottom-up fashion.
    \item \textbf{Sales-Led Growth (SLG) for On-Premise \& Enterprise:} We use a targeted, consultative sales approach to win large customers with complex security and governance needs.
\end{enumerate}
These two engines are designed to be mutually reinforcing: the success of the PLG motion generates leads and brand awareness for the sales team.
